
%%%%%%%%%%%%%%%%%%%%%%%%%%%%%%%%%%%%%%%%%%%%%%%%%%%%%%%%%%%%
\chapter{ZAKLJUČEK}
%%%%%%%%%%%%%%%%%%%%%%%%%%%%%%%%%%%%%%%%%%%%%%%%%%%%%%%%%%%%
%\thispagestyle{fancy}
%Predstavljena formulacija končnih elementov ...

V prvem delu diplomske naloge sem najprej opisal teoretične osnove in predpostavke izračuna z osnovnimi formulami katere uporabljamo pri izračunu energetskih parametrov hidroelektrarne. Podrobno sem opisal postopek izračuna konsumpcijske krivulje za pravokotno, trapezno in strugo poljubne oblike. Po navedenih algoritmih program, ki je nastal v okviru diplomske naloge, tudi izračunava parametre hidroelektrarne. Čeprav so metode izračuna enake tudi za akumulacijske hidroelektrarne, sem se zaradi enostavnosti izračuna omejil le na pretočne hidroelektrarne. V namenjenem času za izdelavo diplomske naloge mi algoritma za optimalno nihanje gladine vode v odvisnosti od potreb električne energije namreč ni uspelo razviti.


V drugem delu sem preveril pravilnost izračunavanja energetskih parametrov hidroelektrarne s programom na namišljenem primeru po metodi za pravokotne in trapezno oblikovane struge in metodi za izračun poljubno oblikovane struge. Rezultate, ki jih je izračunal program sem primerjal z rezultati, ki sem jih izračunal ročno in dokazal, da program računa pravilno. Če primerjamo rezultate izračunane po različnih metodah pa pride do določenega odstopanja zaradi samega matematičnega modela, kar sem tudi prikazal v zadnjem delu poglavja \ref{sec:izracun}.


