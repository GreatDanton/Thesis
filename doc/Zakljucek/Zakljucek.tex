
%%%%%%%%%%%%%%%%%%%%%%%%%%%%%%%%%%%%%%%%%%%%%%%%%%%%%%%%%%%%
\chapter{ZAKLJUČEK}
%%%%%%%%%%%%%%%%%%%%%%%%%%%%%%%%%%%%%%%%%%%%%%%%%%%%%%%%%%%%
%\thispagestyle{fancy}
%Predstavljena formulacija končnih elementov ...

V prvem delu diplomske naloge smo najprej podali teoretične osnove in predpostavke izračuna z osnovnimi formulami, katere uporabljamo pri izračunu energetskih parametrov hidroelektrarne. Podrobno smo opisali postopek izračuna konsumpcijske krivulje za struge pravokotne, trapezne oblike, ter strugo povsem poljubne oblike. Na tej osnovi sem razvil algoritem, ki izračunava energetske parametre pretočne hidroelektrarne.


V drugem delu diplomske naloge smo preverili pravilnost izračunavanja energetskih parametrov hidroelektrarne s programom na namišljenem primeru po metodi za pravokotne in trapezno oblikovane struge ter metodi za izračun poljubno oblikovane struge. Rezultate, ki jih je izračunal program smo primerjali z rezultati, ki smo jih izračunali ročno in dokazali, da program računa pravilno.

 Čeprav so metode izračuna enake tudi za akumulacijske hidroelektrarne, smo se zaradi enostavnosti izračuna omejili le na izračun energetskih parametrov za pretočne hidroelektrarne. Možne smernice za nadaljnje delo so razširitev algoritmov za preračun energetskih parametrov akumulacijskih hidroelektrarn, ki bi v upošteval optimalno nihanje gladine vode v akumulaciji za pregrado v odvisnosti od potreb električne energije.



