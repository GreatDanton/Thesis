
%%%%%%%%%%%%%%%%%%%%%%%%%%%%%%%%%%%%%%%%%%%%%%%%%%%%%%%%%%%%
\chapter{ZAKLJUČEK}
%%%%%%%%%%%%%%%%%%%%%%%%%%%%%%%%%%%%%%%%%%%%%%%%%%%%%%%%%%%%
%\thispagestyle{fancy}
%Predstavljena formulacija končnih elementov ...

V prvem delu diplomske naloge smo najprej opisali teoretične osnove in predpostavke izračuna z osnovnimi formulami katere uporabljamo pri izračunu energetskih parametrov hidroelektrarne. Podrobno smo opisali postopek izračuna konsumpcijske krivulje za pravokotno, trapezno in strugo poljubne oblike. Po navedenih algoritmih program, ki je nastal v okviru diplomske naloge, tudi izračunava energetske parametre hidroelektrarne. Čeprav so metode izračuna enake tudi za akumulacijske hidroelektrarne, smo se zaradi enostavnosti izračuna omejili le na izračun energetskih parametrov za pretočne hidroelektrarne. V namenjenem času za izdelavo diplomske naloge mi algoritma za optimalno nihanje gladine vode v odvisnosti od potreb električne energije namreč ni uspelo razviti.


V drugem delu diplomske naloge smo preverili pravilnost izračunavanja energetskih parametrov hidroelektrarne s programom na namišljenem primeru po metodi za pravokotne in trapezno oblikovane struge ter metodi za izračun poljubno oblikovane struge. Rezultate, ki jih je izračunal program smo primerjali z rezultati, ki smo jih izračunali ročno in dokazali, da program računa pravilno.

