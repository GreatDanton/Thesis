
\chapter{Uvod}\label{sec: Uvod}
\thispagestyle{fancy}


%TODO: citiraj https://www.researchgate.net/profile/Mikos_Matja/publication/248381078_Dragocen_obnovljivi_vir_energije_nam_tece_skozi_prste_hidroelektrarne_na_srednji_Savi/links/54bd428c0cf218d4a16a2483/Dragocen-obnovljivi-vir-energije-nam-tece-skozi-prste-hidroelektrarne-na-srednji-Savi.pdf

Letna količina vode ki se pretoči v Sloveniji je 33,9 $km^{3}$, kar nas primerjano na število prebivalcev uvršča v sam vrh v Evropi, takoj za Švico in Norveško. Potreba po električni energiji se iz leta v leto veča, vendar se le okoli 47\% vodnega potenciala efektivno uporablja za potrebe proizvodnje električne energije. Voda v Sloveniji je povsod okoli nas, zato je zanimivo preračunati koliko električne energije bi lahko proizvedli iz bližnjega potoka ali večje reke. Podatki o pretokih rek v Sloveniji so namreč javno dostopni v arhivu na spletni strani agencije Republike Slovenije za okolje (ARSO). \cite{HEnaSrednjiSavi}


Cilj diplomske naloge je izdelava aplikacije, ki omogoča oceno hidroenergetskega potenciala vodotoka na poljubnem odseku. Vhodni podatek predstavljajo povprečni dnevni pretoki, ki so na voljo iz javno dostopnih podatkovnih baz in parametri rečne struge, ki jih določimo iz razpoložljivih geodetskih podatkov (karakteristični prečni prerezi in naklon struge) in podatkov iz literature (koeficient hrapavosti). Aplikacija omogoča, da s pomočjo začetno ocenjenih parametrov struge in razpoložljivih povprečnih dnevnih pretokov na izbranem odseku vodotoka izračunamo energetske parametre pretočne hidroelektrarne.

%TODO: BEFORE CORRECTION:
%Cilj diplomske naloge je ocena proizvedene električne energije pretočne hidroelektrarne poljubne velikosti za vodotok poljubnega pretoka in oblike. V ta namen sem napisal program, ki s pomočjo začetno ocenjenih parametrov struge in meritev povprečnih dnevnih pretokov izbranega vodotoka izračuna energetske parametre pretočne hidroelektrarne. 


V diplomski nalogi bom najprej opisal postopek izračuna z osnovnimi enačbami, na koncu pa bom primerjal rezultate ročnega izračuna z rezultati ki jih izračuna program. Pri izračunu sem upošteval da je voda za pregrado na maksimalni konstantni višini, izkoristek turbine konstanten in neodvisen od pretoka skozi turbino ter naklon struge od 0\% do 2\%.



%-----------------------------------------------------------